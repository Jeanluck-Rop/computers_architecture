\documentclass[12pt,letterpaper]{article}
\usepackage[utf8]{inputenc}
\usepackage{listings, float, xcolor}

%----- Configuración del estilo del documento------%
\usepackage{graphicx, fancyhdr, lastpage}
\usepackage{enumitem, pifont, hyperref, ulem}
\usepackage{colortbl, tikz}
\usetikzlibrary{matrix,calc}
\usepackage[left=2cm,right=2cm,top=1.8cm,bottom=2.3cm]{geometry}

%------ Paquetes matemáticos básicos --------%
\usepackage{amsmath, amssymb, amsthm}

%------ Personalizar el link al video  --------%
\hypersetup{
  colorlinks=true,
  linkcolor=blue!50!black, % Azul oscuro
  urlcolor=blue!50!black,  % Azul oscuro
  hidelinks % Elimina el recuadro azul
}

\renewcommand{\refname}{Referencias}

\definecolor{mostaza}{rgb}{1,0.85,0.4}
\definecolor{dblue}{rgb}{0.0, 0.2, 0.6}

\newcommand{\ekis}{\textcolor{dblue}{\ding{55}}}

% === DEFINICIONES PARA MAPAS DE KARNAUGH === %

%isolated term
%#1 - Optional. Space between node and grouping line. Default=0
%#2 - node
%#3 - filling color
\newcommand{\implicantsol}[3][0]{
  \draw[rounded corners=3pt, fill=#3, opacity=0.3] ($(#2.north west)+(135:#1)$) rectangle ($(#2.south east)+(-45:#1)$);
}

%internal group
%#1 - Optional. Space between node and grouping line. Default=0
%#2 - top left node
%#3 - bottom right node
%#4 - filling color
\newcommand{\implicant}[4][0]{
  \draw[rounded corners=3pt, fill=#4, opacity=0.3] ($(#2.north west)+(135:#1)$) rectangle ($(#3.south east)+(-45:#1)$);
}

%group lateral borders
%#1 - Optional. Space between node and grouping line. Default=0
%#2 - top left node
%#3 - bottom right node
%#4 - filling color
\newcommand{\implicantcostats}[4][0]{
  \draw[rounded corners=3pt, fill=#4, opacity=0.3] ($(rf.east |- #2.north)+(90:#1)$)-| ($(#2.east)+(0:#1)$) |- ($(rf.east |- #3.south)+(-90:#1)$);
  \draw[rounded corners=3pt, fill=#4, opacity=0.3] ($(cf.west |- #2.north)+(90:#1)$) -| ($(#3.west)+(180:#1)$) |- ($(cf.west |- #3.south)+(-90:#1)$);
}

%group top-bottom borders
%#1 - Optional. Space between node and grouping line. Default=0
%#2 - top left node
%#3 - bottom right node
%#4 - filling color
\newcommand{\implicantdaltbaix}[4][0]{
  \draw[rounded corners=3pt, fill=#4, opacity=0.3] ($(cf.south -| #2.west)+(180:#1)$) |- ($(#2.south)+(-90:#1)$) -| ($(cf.south -| #3.east)+(0:#1)$);
  \draw[rounded corners=3pt, fill=#4, opacity=0.3] ($(rf.north -| #2.west)+(180:#1)$) |- ($(#3.north)+(90:#1)$) -| ($(rf.north -| #3.east)+(0:#1)$);
}

%group corners
%#1 - Optional. Space between node and grouping line. Default=0
%#2 - filling color
\newcommand{\implicantcantons}[2][0]{
  \draw[rounded corners=3pt, opacity=.3] ($(rf.east |- 0.south)+(-90:#1)$) -| ($(0.east |- cf.south)+(0:#1)$);
  \draw[rounded corners=3pt, opacity=.3] ($(rf.east |- 8.north)+(90:#1)$) -| ($(8.east |- rf.north)+(0:#1)$);
  \draw[rounded corners=3pt, opacity=.3] ($(cf.west |- 2.south)+(-90:#1)$) -| ($(2.west |- cf.south)+(180:#1)$);
  \draw[rounded corners=3pt, opacity=.3] ($(cf.west |- 10.north)+(90:#1)$) -| ($(10.west |- rf.north)+(180:#1)$);
  \fill[rounded corners=3pt, fill=#2, opacity=.3] ($(rf.east |- 0.south)+(-90:#1)$) -|  ($(0.east |- cf.south)+(0:#1)$) [sharp corners] ($(rf.east |- 0.south)+(-90:#1)$) |-  ($(0.east |- cf.south)+(0:#1)$) ;
  \fill[rounded corners=3pt, fill=#2, opacity=.3] ($(rf.east |- 8.north)+(90:#1)$) -| ($(8.east |- rf.north)+(0:#1)$) [sharp corners] ($(rf.east |- 8.north)+(90:#1)$) |- ($(8.east |- rf.north)+(0:#1)$) ;
  \fill[rounded corners=3pt, fill=#2, opacity=.3] ($(cf.west |- 2.south)+(-90:#1)$) -| ($(2.west |- cf.south)+(180:#1)$) [sharp corners]($(cf.west |- 2.south)+(-90:#1)$) |- ($(2.west |- cf.south)+(180:#1)$) ;
  \fill[rounded corners=3pt, fill=#2, opacity=.3] ($(cf.west |- 10.north)+(90:#1)$) -| ($(10.west |- rf.north)+(180:#1)$) [sharp corners] ($(cf.west |- 10.north)+(90:#1)$) |- ($(10.west |- rf.north)+(180:#1)$) ;
}

%Empty Karnaugh map 4x4
\newenvironment{Karnaugh}%
{
  \begin{tikzpicture}[baseline=(current bounding box.north),scale=0.8]
    \draw (0,0) grid (4,4);
    \draw (0,4) -- node [pos=0.7,above right,anchor=south west] {$x_2x_3$} node [pos=0.7,below left,anchor=north east] {$x_0x_1$} ++(135:1);
    %
    \matrix (mapa) [matrix of nodes,
      column sep={0.8cm,between origins},
      row sep={0.8cm,between origins},
      every node/.style={minimum size=0.3mm},
      anchor=8.center,
      ampersand replacement=\&] at (0.5,0.5)
            {
              \& |(c00)| 00         \& |(c01)| 01         \& |(c11)| 11         \& |(c10)| 10         \& |(cf)| \phantom{00} \\
              |(r00)| 00             \& |(0)|  \phantom{0} \& |(1)|  \phantom{0} \& |(3)|  \phantom{0} \& |(2)|  \phantom{0} \&                     \\
              |(r01)| 01             \& |(4)|  \phantom{0} \& |(5)|  \phantom{0} \& |(7)|  \phantom{0} \& |(6)|  \phantom{0} \&                     \\
              |(r11)| 11             \& |(12)| \phantom{0} \& |(13)| \phantom{0} \& |(15)| \phantom{0} \& |(14)| \phantom{0} \&                     \\
              |(r10)| 10             \& |(8)|  \phantom{0} \& |(9)|  \phantom{0} \& |(11)| \phantom{0} \& |(10)| \phantom{0} \&                     \\
              |(rf) | \phantom{00}   \&                    \&                    \&                    \&                    \&                     \\
            };
}%
{
  \end{tikzpicture}
}

%Empty Karnaugh map 2x4
\newenvironment{Karnaughvuit}%
{
  \begin{tikzpicture}[baseline=(current bounding box.north),scale=0.8]
    \draw (0,0) grid (4,2);
    \draw (0,2) -- node [pos=0.7,above right,anchor=south west] {$yz$} node [pos=0.7,below left,anchor=north east] {$x$} ++(135:1);
    %
    \matrix (mapa) [matrix of nodes,
      column sep={0.8cm,between origins},
      row sep={0.8cm,between origins},
      every node/.style={minimum size=0.3mm},
      anchor=4.center,
      ampersand replacement=\&] at (0.5,0.5)
            {
              \& |(c00)| 00         \& |(c01)| 01         \& |(c11)| 11         \& |(c10)| 10         \& |(cf)| \phantom{00} \\
              |(r00)| 0             \& |(0)|  \phantom{0} \& |(1)|  \phantom{0} \& |(3)|  \phantom{0} \& |(2)|  \phantom{0} \&                     \\
              |(r01)| 1             \& |(4)|  \phantom{0} \& |(5)|  \phantom{0} \& |(7)|  \phantom{0} \& |(6)|  \phantom{0} \&                     \\
              |(rf) | \phantom{00}  \&                    \&                    \&                    \&                    \&                     \\
            };
}%
{
  \end{tikzpicture}
}

%Empty Karnaugh map 2x2
\newenvironment{Karnaughquatre}%
{
  \begin{tikzpicture}[baseline=(current bounding box.north),scale=0.8]
    \draw (0,0) grid (2,2);
    \draw (0,2) -- node [pos=0.7,above right,anchor=south west] {b} node [pos=0.7,below left,anchor=north east] {a} ++(135:1);
    %
    \matrix (mapa) [matrix of nodes,
      column sep={0.8cm,between origins},
      row sep={0.8cm,between origins},
      every node/.style={minimum size=0.3mm},
      anchor=2.center,
      ampersand replacement=\&] at (0.5,0.5)
            {
              \& |(c00)| 0          \& |(c01)| 1  \\
              |(r00)| 0 \& |(0)|  \phantom{0} \& |(1)|  \phantom{0} \\
              |(r01)| 1 \& |(2)|  \phantom{0} \& |(3)|  \phantom{0} \\
            };
}%
{
  \end{tikzpicture}
}

%Empty Karnaugh map 4x8
\newenvironment{KarnaughTrentaDos}%
{
  \begin{tikzpicture}[baseline=(current bounding box.north),scale=0.8]
    \draw (0,0) grid (8,4); % 8 columnas, 4 filas
    \draw (0,4) -- node [pos=0.7,above right,anchor=south west] {$x_2x_3x_4$} node [pos=0.7,below left,anchor=north east] {$x_0x_1$} ++(135:1);
    %
    \matrix (mapa) [matrix of nodes,
      column sep={0.8cm,between origins},
      row sep={0.8cm,between origins},
      every node/.style={minimum size=0.3mm},
      anchor=16.center,
      ampersand replacement=\&] at (0.5,0.5)
            {
              \& |(c000)| 000         \& |(c001)| 001         \& |(c011)| 011         \& |(c010)| 010         \& |(c110)| 110         \& |(c111)| 111         \& |(c101)| 101         \& |(c100)| 100         \& |(cf)| \phantom{000}  \\ 
              |(r000)| 000            \& |(0)|    \phantom{0} \& |(1)|    \phantom{0} \& |(3)|    \phantom{0} \& |(2)|    \phantom{0} \& |(6)|    \phantom{0} \& |(7)|    \phantom{0} \& |(5)|    \phantom{0} \& |(4)|  \phantom{0} \& \\
              |(r001)| 001            \& |(8)|    \phantom{0} \& |(9)|    \phantom{0} \& |(11)|   \phantom{0} \& |(10)|   \phantom{0} \& |(14)|   \phantom{0} \& |(15)|   \phantom{0} \& |(13)|   \phantom{0} \& |(12)| \phantom{0} \& \\
              |(r011)| 011            \& |(24)|   \phantom{0} \& |(25)|   \phantom{0} \& |(27)|   \phantom{0} \& |(26)|   \phantom{0} \& |(30)|   \phantom{0} \& |(31)|   \phantom{0} \& |(29)|   \phantom{0} \& |(28)| \phantom{0} \& \\
              |(r010)| 010            \& |(16)|   \phantom{0} \& |(17)|   \phantom{0} \& |(19)|   \phantom{0} \& |(18)|   \phantom{0} \& |(22)|   \phantom{0} \& |(23)|   \phantom{0} \& |(21)|   \phantom{0} \& |(20)| \phantom{0} \& \\
              |(rf)| \phantom{000}    \&   \&   \&   \&   \&   \&   \&   \&   \& \\
            };
}%
{
  \end{tikzpicture}
}

%Defines 8 or 16 values (0,1,X)
\newcommand{\contingut}[1]{%
  \foreach \x [count=\xi from 0]  in {#1}
  \path (\xi) node {\x};
}

%Places 1 in listed positions
\newcommand{\minterms}[1]{%
  \foreach \x in {#1}
  \path (\x) node {1};
}

%Places 0 in listed positions
\newcommand{\maxterms}[1]{%
  \foreach \x in {#1}
  \path (\x) node {0};
}

%Places X in listed positions
\newcommand{\indeterminats}[1]{%
  \foreach \x in {#1}
  \path (\x) node {X};
}

\begin{document}

%------ Encabezado -------- %
\bigskip
\hrule height 0.1pt
\bigskip

\begin{center}
  \begin{minipage}{3cm}
    \begin{center}
      \includegraphics[height=3.4cm]{../unam_logo.png}
    \end{center}
  \end{minipage}\hfill
  \begin{minipage}{10cm}
    \begin{center}
      \textbf{\Large Universidad Nacional Autónoma de México}\\[0.2cm]
      \textbf{\large Facultad de Ciencias}\\[0.2cm]
      \textbf{Organización y Arquitectura de Computadoras 2025-2}\\[0.4cm]
      \textbf{\Large Tarea 03}\\[0.1cm]
      \textbf{Docentes:}\\
      José Galaviz \hspace{1em} Ricardo Pérez \hspace{1em} Ximena Lezama\\[0.3cm]
      \textbf{Autores:}\\
      Fernanda Ramírez Juárez \quad Ianluck Rojo Peña\\[0.2cm]
      \textbf{No. de cuenta:}\\
      321204747 \quad 118005762\\[0.2cm]
      \textbf{Fecha de entrega:} Viernes 11 de abril de 2025
    \end{center}
  \end{minipage}\hfill
  \begin{minipage}{3cm}
    \begin{center}
      \includegraphics[height=3.4cm]{../fc_logo.png}
    \end{center}
  \end{minipage}
\end{center}

\bigskip
\hrule height 0.1pt
\bigskip

%------ Contenido -------- %
\section*{Preguntas.}

\begin{enumerate}[label=\arabic*.]
\item Demuestra que $x \cdot (y \cdot z) = (x \cdot y) \cdot z$\\
  % -- Respuesta -- %
  \textbf{Dem.} Demostracion por tabla de verdad:
  
  \begin{center}
    \begin{tabular}{| c | c | c | c | c | c | c |}
      \hline
      \textbf{$x$} & \textbf{$y$} & \textbf{$z$} & \textbf{$y \cdot z$} & \textbf{$x \cdot (y \cdot z)$} & \textbf{$x \cdot y$} & \textbf{$(x \cdot y) \cdot z$} \\
      \hline
      0 & 0 & 0 & 0 & 0 & 0 & 0 \\
      0 & 0 & 1 & 0 & 0 & 0 & 0 \\
      0 & 1 & 0 & 0 & 0 & 0 & 0 \\ 
      0 & 1 & 1 & 1 & 0 & 0 & 0 \\
      1 & 0 & 0 & 0 & 0 & 0 & 0 \\
      1 & 0 & 1 & 0 & 0 & 0 & 0 \\
      1 & 1 & 0 & 0 & 0 & 1 & 0 \\
      1 & 1 & 1 & 1 & 1 & 1 & 1 \\
      \hline
    \end{tabular}
  \end{center}

  Como las columnas $x \cdot (y \cdot z)$ y $(x \cdot y) \cdot z$ son idénticas para todas las combinaciones posibles de $x$, $y$, $z$, queda demostrada la igualdad.
  \bigskip
  
\item Demuestra si la siguiente igualdad es válida $x(\overline{x} + y) = xy$\\
  % -- Respuesta -- %
  \textbf{Dem.}

  Aplicamos la propiedad de distributividad:
  \[
  x(\overline{x} + y) = x\overline{x} + xy
  \]

  Por el complemento, tenemos que $x\overline{x} = 0$, as\'{i}:
  \[
  x\overline{x} + xy = 0 + xy
  \]

  Y por \'{u}ltimo por Identidad $0 + xy = xy + 0 = xy$:
  \[
  0 + xy = xy
  \]

  Por lo tanto se cumple la igualdad.

  Alternativamente con la tabla de verdad:
  \begin{center}
    \begin{tabular}{| c | c | c | c | c | c |}
      \hline
      \textbf{$x$} & \textbf{$y$} & \textbf{$\overline{x}$} & \textbf{$\overline{x} + y$} & \textbf{$x (\overline{x} + y)$} & \textbf{$xy$} \\
      \hline
      0 & 0 & 1 & 1 & 0 & 0 \\
      0 & 1 & 1 & 1 & 0 & 0 \\ 
      1 & 0 & 0 & 0 & 0 & 0 \\
      1 & 1 & 0 & 1 & 1 & 1 \\
      \hline
    \end{tabular}
  \end{center}

  Las columnas $x(\overline{x} + y)$ y $xy$ son iguales, confirmando la igualdad.
  \bigskip
  
\item Demuestra si la siguiente igualdad es válida $(x + y)(\overline{x} + z)(y + z) = (x + y)(\overline{x} + z)$ \\
  % -- Respuesta -- %
  \textbf{Dem.}
  
  Tomemos el lado izquierdo $(x + y)(\overline{x} + z)(y + z)$ y desarrollemos el producto de los dos primeros t\'{e}rminos $(x + y)(\overline{x} + z)$:
  \[
  (x + y)(\overline{x} + z) = x\overline{x} + xz + y\overline{x} + yz
  \]

  Sabemos que por propiedad del complemento $x\overline{x} = 0$:
  \[
  x\overline{x} + xz + y\overline{x} + yz = 0 xz + y\overline{x} + yz = xz + \overline{x}y + yz
  \]

  Es decir:
  \[
  (x + y)(\overline{x} + z) = xz + \overline{x}y + yz
  \]
  
  Retomando en el lado izquierdo:
  \[
  (x + y)(\overline{x} + z)(y + z) = (xz + \overline{x}y + yz)(y + z)
  \]

  Desarrollando el producto:
  \begin{align*}
    (xz + y\overline{x} + yz)(y + z) &= xzy + y\overline{x}y + yzy + xzz + y\overline{x}z + yzz \\
    &= xyz + \overline{x}yy + yyz + xzz + \overline{x}yz + yzz \quad \text{Conmutatividad}
  \end{align*}

  Aplicando idempotencia, $yy = y$ y $zz = z$:
  \[
  xyz + \overline{x}yy + yyz + xzz + \overline{x}yz + yzz = xyz + \overline{x}y + yz + xz + \overline{x}yz + yz
  \]

  De igual forma por idempotencia sobre $yz + yz = yz$
  \begin{align*}
    xyz + \overline{x}y + yz + xz + \overline{x}yz + yz &= xyz + \overline{x}y + xz + \overline{x}yz + yz + yz \quad \text{Conmutatividad}  \\
    &= xyz + \overline{x}y + xz + \overline{x}yz + yz
  \end{align*}
  
  Observemos que $xyz + \overline{x}yz = yz(x + \overline{x}) = yz(1) = yz$
  \begin{align*}
    xyz + \overline{x}y + xz + \overline{x}yz + yz &= xyz + \overline{x}yz + \overline{x}y + xz + yz \hspace{1.85cm} \text{Conmutatividad} \\
    &= \overline{x}y + xz + yz + yz = \overline{x}y + xz + yz \quad \text{Idempotencia}
  \end{align*}

  Ahora, notemos que $\overline{x}y + xz + yz = xz + \overline{x}y + yz$, que es lo que obtuvimos al desarrollar el producto $(x + y)(\overline{x} + z)$, es decir:
  \[
  \overline{x}y + xz + yz = xz + \overline{x}y + yz = (x + y)(\overline{x} + z)
  \]

  Por lo que podemos concluir que el factor $(y + z)$ est\'{a} de m\'{a}s, y de esta forma la igualdad es v\'{a}lida.

  \[
  \therefore \; \text{ Se cumple que } (x + y)(\overline{x} + z)(y + z) = (x + y)(\overline{x} + z)
  \]
  \bigskip
  
\item Demuestra si la siguiente igualdad es válida $\overline{x \cdot y} = \overline{x} \cdot \overline{y}$\\
  % -- Respuesta -- %
  \textbf{Dem.} Demostracion por contra-ejemplo:
  
  Sea $x = 1$ y $y = 1$, sustituyendo en la igualdad:
  \[
  \overline{x \cdot y} = \overline{x} \cdot \overline{y}
  \]
  \[
  \overline{1 \cdot 0} = \overline{1} \cdot \overline{0}
  \]

  Sabemos que $1 \cdot 0 = 0 \cdot 1 = 0$, y $\overline{1} = 0$, $\overline{0} = 1$:
  \[
  \overline{0} = 0 \cdot 1
  \]
  \[
  1 = 0 \text{\quad \; \large{!!!}}
  \]

  Pero esto es una contradicci\'{o}n, pues $1 \neq 0$. Por lo que la igualdad no es válida.
  \[
  \therefore \quad \overline{x \cdot y} \neq \overline{x} \cdot \overline{y}
  \]
  
  \bigskip
  
\item Verifica la siguiente igualdad usando los postulados de Huntington.
  \[
  F(x, y, z) = x + x(\overline{x} + y) + \overline{x}y = x + y
  \]
  % -- Respuesta -- %
  Resolvemos la igualdad en base a los postulados de Huntington tomados como referencia del archivo pdf de las notas de clase \href{https://drive.google.com/file/d/1BdCwuwFcSar5W5nPxA98FVNfTxZfd6yg/view}{\textit{\uline{Lógica digital y diseño de circuitos digitales}}}.\\

  Aplicamos la distributividad para $x(\overline{x}\: +\: y) = x\overline{x}\: +\: xy$:
  \begin{align*}
    F(x, y, z) = x + x(\overline{x} + y) + \overline{x}y =  x + x\overline{x} + xy + \overline{x}y
  \end{align*}

  Notemos que, por el postulado Complemento, $x\overline{x} = 0$:
  \begin{align*}
    F(x, y, z) = x + x\overline{x} + xy + \overline{x}y = x + 0 + xy + \overline{x}y
  \end{align*}

  Y por Identidad $x + 0 = 0 + x = x$:
  \begin{align*}
    F(x, y, z) =  x + 0 + xy + \overline{x}y = x + xy + \overline{x}y
  \end{align*}

  Ahora, podemos factorizar $y$ de $(xy + \overline{x}y) = y(x + \overline{x})$:
  \begin{align*}
    F(x, y, z) = x + xy + \overline{x}y = x + y(x + \overline{x})
  \end{align*}

  Reexpresamos $x + \overline{x} = 1$ por el postulado del Complemento:
  \begin{align*}
    F(x, y, z) = x + y(x + \overline{x}) = x + y(1) = x + y \cdot 1
  \end{align*}

  Simplificamos $y \cdot 1 = y$ por el postulado de Identidad:
  \begin{align*}
    F(x, y, z) =  x + y(1) = x + y \cdot 1 = x + y
  \end{align*}

  De este modo qued\'{o} demostrado que $x + x(\overline{x} + y) + \overline{x}y = x + y$
  \[
  \therefore \; F(x, y, z) = x + x(\overline{x} + y) + \overline{x}y = x + y
  \]
  \bigskip
  
\item Obtén los mint\'{e}rminos y reduce la siguiente función.
  \[
  F(x, y, z) = \overline{x} \cdot \overline{y} \cdot \overline{z} \cdot x + \overline{z} \cdot x + z \cdot x + x \cdot \overline{y} + \overline{z}
  \]
  % -- Respuesta -- %
  Antes de obtener los mint\'{e}rminos y reducir la funci\'{o}n dada, primero la simplificamos en base, de nuevo, a las notas de clase \href{https://drive.google.com/file/d/1BdCwuwFcSar5W5nPxA98FVNfTxZfd6yg/view}{\textit{\uline{Lógica digital y diseño de circuitos digitales}}}.\\

  \begin{enumerate}[label=\arabic*)]
  \item Simplificamos la funci\'{o}n.

    Notemos que por Conmutatividad $\overline{x} \cdot \overline{y} \cdot \overline{z} \cdot x = \overline{x} \cdot x \cdot \overline{y} \cdot \overline{z}$
    
    Por Complemento $\overline{x} \cdot x = 0$, as\'{i} $\overline{x} \cdot x \cdot \overline{y} \cdot \overline{z} = 0 \cdot \overline{y} \cdot \overline{z}$
    
    Y por el \textbf{Teorema 2} (Aniquilaci\'{o}n) $0 \cdot \overline{y} \cdot \overline{z} = 0$:
    \begin{align*}
      F(x, y, z) &= \overline{x} \cdot \overline{y} \cdot \overline{z} \cdot x + \overline{z} \cdot x + z \cdot x + x \cdot \overline{y} + \overline{z}\\
      &= \overline{x} \cdot x \cdot \overline{y} \cdot \overline{z} + \overline{z} \cdot x + z \cdot x + x \cdot \overline{y} + \overline{z}\\
      &= 0 \cdot \overline{y} \cdot \overline{z} + \overline{z} \cdot x + z \cdot x + x \cdot \overline{y} + \overline{z}\\
      &= 0 + \overline{z} \cdot x + z \cdot x + x \cdot \overline{y} + \overline{z}\\
      &= \overline{z} \cdot x + z \cdot x + x \cdot \overline{y} + \overline{z} \quad \text{Identidad: } 0 + x = x
    \end{align*}

    Factorizamos los términos comunes, observando que $\overline{z} \cdot x + z \cdot x = x(\overline{z} + z)$

    Recordando que por el Complemento $\overline{z} + z = 1$, as\'{i} $x(\overline{z} + z) = x(1) = x$:
    \begin{align*}
      F(x, y, z) &= \overline{z} \cdot x + z \cdot x + x \cdot \overline{y} + \overline{z}\\
      &= x(\overline{z} + z) + x \cdot \overline{y} + \overline{z}\\
      &= x(1) + x \cdot \overline{y} + \overline{z}\\
      &= x + x \cdot \overline{y} + \overline{z}
    \end{align*}

    Por \'{u}ltimo aplicamos el \textbf{Teorema 3} (Absorci\'{o}n) $x + x \cdot y = x$:
    \begin{align*}
      F(x, y, z) &=  x + x \cdot \overline{y} + \overline{z}\\
      &= x + \overline{z}
    \end{align*}

    Por lo que la funci\'{o}n simplificada es $F(x, y, z) = x + \overline{z}$.

  \item Obtenemos los mint\'{e}rminos.
    
    Construimos la tabla de verdad para la función $F(x, y, z) = x + \overline{z}$:
    \begin{center}
      \begin{tabular}{|c|c|c|c|c|}
        \hline
        $x$ & $y$ & $z$ & $\overline{z}$ & $F(x, y, z)$ \\
        \hline
        0 & 0 & 0 & 1 & 1 \\
        0 & 0 & 1 & 0 & 0 \\
        0 & 1 & 0 & 1 & 1 \\
        0 & 1 & 1 & 0 & 0 \\
        1 & 0 & 0 & 1 & 1 \\
        1 & 0 & 1 & 0 & 1 \\
        1 & 1 & 0 & 1 & 1 \\
        1 & 1 & 1 & 0 & 1 \\
        \hline
      \end{tabular}
    \end{center}
    
    Los mint\'{e}rminos donde $F = 1$ corresponden a las siguientes combinaciones:
    \[
    m_0, m_2, m_4, m_5, m_6, m_7
    \]
    
    Y de este modo, la forma canónica es:
    \[
    F(x, y, z) = m_0, m_2, m_4, m_5, m_6, m_7 = \sum m(0, 2, 4, 5, 6, 7)
    \]

  \end{enumerate}
  
  \[
  \therefore \text{ La funci\'{o}n reducida es } F(x, y, z) = x + \overline{z}
  \]
  \[
  \quad \;   \text{ Los mint\'{e}rminos son } F(x, y, z) = m_0, m_2, m_4, m_5, m_6, m_7 = \sum m(0, 2, 4, 5, 6, 7)
  \]
  
  \newpage
  
\item Simplifica la siguiente función usando su tabla de verdad asociada y mapas de Karnaugh.
  \[
  F(x, y, z) = \overline{xyz} + \overline{xy}z + \overline{x}y\overline{z} + x\overline{yz} + \overline{x}yz + x\overline{y}z + xyz
  \]
  % -- Respuesta -- %
  
  \textbf{Tabla de verdad:}
  \[
  \begin{array}{| c | c | c | c |}
    \hline
    x & y & z & xyz \\
    \hline
    0 & 0 & 0 & 1 \\
    0 & 0 & 1 & 1 \\
    0 & 1 & 0 & 1 \\
    0 & 1 & 1 & 1 \\
    1 & 0 & 0 & 1 \\
    1 & 0 & 1 & 1 \\
    1 & 1 & 0 & 0 \\
    1 & 1 & 1 & 1 \\
    \hline
  \end{array}
  \]

  \textbf{Mapa de Karnaugh:}
  \begin{Karnaughvuit}
    \contingut{1,1,1,1,1,1,1,0}

    \implicant{0}{5}{orange}     % 0,1,4,5
    \implicant{1}{7}{purple}     % 1,3,5,7
    \implicant{0}{2}{blue}       % 0,1,2,3
  \end{Karnaughvuit}
  
  \textbf{Función simplificada:} $\overline{x} + \overline{y} + z$
  
  \bigskip
  
\item Reduce la siguiente función y da su maxitérminos.
  \[
  F(x, y, z) = (x + \overline{x}z) \cdot (\overline{y} + \overline{z}) \cdot z
  \]
  % -- Respuesta -- %

  Aplicamos la ley de absorción y la ley distributiva:
  \[
  x + \overline{x}z = (x + \overline{x})(x + z) = 1 \cdot (x + z) = x + z
  \]
  
  Sustituimos:
  \[
  F(x, y, z) = (x + z)(\overline{y} + \overline{z})z
  \]
  
  Aplicamos la ley distributiva entre los tres factores:
  \[
  F(x, y, z) = [(x + z)(\overline{y} + \overline{z})]z
  \]

  Distribuimos los primeros términos:
  \[
  (x + z)(\overline{y} + \overline{z}) = x\overline{y} + x\overline{z} + z\overline{y} + z\overline{z}
  \]

  Pero $z\overline{z} = 0$, entonces obtenemos: $x\overline{y} + x\overline{z} + z\overline{y}$

  Multiplicamos por $z$:
  \[
  F(x, y, z) = (x\overline{y} + x\overline{z} + z\overline{y})z = x\overline{y}z + x\overline{z}z + z\overline{y}z
  \]
  
  Como ya sabemos $x\overline{z}z = 0$ y $z\overline{y}z = z\overline{y}$:
  \[
  F(x, y, z) = x\overline{y}z + z\overline{y}
  \]

  Factorizando:
  \[
  F(x, y, z) = z\overline{y}(x + 1) = z\overline{y}
  \]

  As\'{i} tenemos la \textbf{Función reducida:} $z\overline{y}$

  \textbf{Escribimos cada maxitérmino:}
  \[
  \begin{array}{cccc}
  M_0 & 000 & (x + y + z) \\
  M_1 & 001 & (x + y + \overline{z}) \\
  M_2 & 010 & (x + \overline{y} + z) \\
  M_3 & 011 & (x + \overline{y} + \overline{z}) \\
  M_4 & 100 & (\overline{x} + y + z) \\
  M_6 & 110 & (\overline{x} + \overline{y} + z) \\
  M_7 & 111 & (\overline{x} + \overline{y} + \overline{z}) \\
  \end{array}
  \]

  \[
  F(x, y, z) = \prod(M_0, M_1, M_2, M_3, M_4, M_6, M_7)
  \]
  
  \[
  F(x, y, z) = (x + y + z)(x + \overline{y} + z)(x + \overline{y} + \overline{z})(\overline{x} + y + z)(\overline{x} + \overline{y} + z)(\overline{x} + \overline{y} + \overline{z})
  \]
  
  \[
  \boxed{F(x, y, z) = (000)(010)(011)(100)(110)(111)}
  \]
  
  \bigskip
  
\item Utilizando Mapas de Karnaugh simplifica la función.
  \[
  \centering
  F(x_0, x_1, x_2, x_3) = \overline{x_0x_1x_2x_3} + \overline{x_0x_1x_2}x_3 + \overline{x_0x_1}x_2x_3 + x_0\overline{x_1}x_2x_3 + x_0x_1\overline{x_2x_3} + \overline{x_0}x_1\overline{x_2x_3} + x_0x_1x_2x_3
  \]
  % -- Respuesta -- %
  
  \textbf{Mapa de Karnaugh:}
  \begin{Karnaugh}
    % Contenido del mapa: (1 = celda activa, 0 = celda inactiva)
    % Orden de índices: 0, 1, 3, 2, 4, 5, 7, 6, 12, 13, 15, 14, 8, 9, 11, 10
    \contingut{1,1,1,0,1,0,0,0,1,0,1,0,0,0,0,0}

    \implicant{0}{1}{red}       % Agrupa celdas 0 y 1
    \implicant{1}{3}{cyan}      % Agrupa celdas 1 y 3
    \implicant{4}{12}{violet}   % Agrupa celdas 4 y 12
    \implicant{15}{11}{pink}     % Agrupa celdas 12 y 15 (index 10 = celda 15 en orden canónico)
  \end{Karnaugh}
  
  \[
  \therefore \text{La \textbf{Funci\'{o}n simplificada} es:}
  \]
  \[
  F(x_0, x_1, x_2, x_3) = x_1\overline{x_2x_3} + \overline{x_0x_1x_2} + \overline{x_0x_1}x_2x_3 + x_0x_2x_3
  \]
  
  \bigskip
  
\item Para realizar una Mapa de Karnaugh con más de 5 variables se mencionó que existe más de una forma de representarlo.
  
  Investiga ambos métodos y utiliza el que más se te acomode para reducir la siguiente función.
  \[
  F(x_0, x_1, x_2, x_3, x_4) = \overline{x_0x_1x_2x_3x_4}\; +\; \overline{x_0x_1x_2}x_3\overline{x_4}\; +\; \overline{x_0x_1}x_2x_3\overline{x_4}\; +\; x_0\overline{x_1}x_2x_3x_4\; +
  \]
  \[
  x_0x_1\overline{x_2x_3}x_4\; +\; \overline{x_0}x_1\overline{x_2x_3x_4}\; +\; x_0x_1x_2x_3x_4
  \]
  % -- Respuesta -- %

  \textbf{Mapa de Karnaugh:}
  \begin{KarnaughTrentaDos}
    % Contenido del mapa: (1 = celda activa, 0 = celda inactiva)
    % Orden de índices: 0, 1, 2, 3, 4, 5, 6, 7, 8, 9, 10, 11, 12, 13, 14, 15, 16, 17, 18, 19, 20, 21, 22, 23, 24, 25, 26, 27, 28, 29, 30, 31
    \contingut{1,0,1,0,0,0,1,0,0,1,0,0,0,0,0,0,0,0,0,0,0,0,0,1,0,1,0,0,0,0,0,1}
    %\contingut{0,1,2,3,4,5,6,7,8,9,10,11,12,13,14,15,16,17,18,19,20,21,22,23,24,25,26,27,28,29,30,31}
    %\contingut{
    %  0,  1,  3,  2,  6,  7,  5,  4,
    %  8,  9, 11, 10, 14, 15, 13, 12,
    % 24, 25, 27, 26, 30, 31, 29, 28,
    % 16, 17, 19, 18, 22, 23, 21, 20,
    %}
    \implicant{0}{0}{red}
    \hspace{-0.1cm}\implicant{2}{2}{red}
    \hspace{0.1cm}\implicant{9}{25}{green}    
    \implicant{2}{6}{cyan}
    \implicant{31}{23}{violet}
  \end{KarnaughTrentaDos}
  
  \[
  \therefore \text{La \textbf{Funci\'{o}n simplificada} es:}
  \]
  \[
  F(x_0, x_1, x_2, x_3) = x_1\overline{x_2x_3} + \overline{x_0x_1x_2} + \overline{x_0x_1}x_2x_3 + x_0x_2x_3
  \]
  
  \bigskip
  
\item Utilizando el algoritmo de Quine–McCluskey realiza la siguiente reducción.
  \[
  F(x_0, x_1, x_2, x_3, x_4) = \overline{x_0x_1x_2x_3x_4}\; +\; \overline{x_0x_1x_2}x_3\overline{x_4}\; +\; \overline{x_0}x_1x_2x_3\overline{x_4}\; +\; \overline{x_0}x_1\overline{x_2}x_3\overline{x_4}\; +
  \]
  \[
  x_0\overline{x_1}x_2x_3x_4\; +\; x_0x_1\overline{x_2x_3}x_4\; +\; \overline{x_0}x_1\overline{x_2x_3}x_4\; +\; x_0x_1x_2\overline{x_3}x_4\; +\; x_0x_1x_2x_3x_4
  \]
  % -- Respuesta -- %

  \begin{enumerate}[label=\arabic*)]
  \item Paso 1: Hacer la tabla y organizar según su índice.
    \[
    \begin{array}{|c|c|c|}
      \hline
      \text{Número} & x_0\,x_1\,x_2\,x_3\,x_4 & \text{Índice} \\
      \hline
      0  & 00000 & 0 \\
      2  & 00010 & 1 \\
      9  & 01001 & 2 \\
      10 & 01010 & 2 \\
      14 & 01110 & 3 \\
      25 & 11001 & 3 \\
      23 & 10111 & 4 \\
      29 & 11101 & 4 \\
      31 & 11111 & 5 \\
      \hline
    \end{array}
    \]
    
  \item Paso 2: Identificar cuáles cambian en un solo bit.
    \[
    \begin{array}{| c | c | c | c |}
      \hline
      \text{Combinación} & x_0\,x_1\,x_2\,x_3\,x_4 & \text{Índice} & \text{Solución} \\
      \hline
      0{-}2   & 000{-}0 & 0 & S_1 \\
      2{-}10  & 0{-}010 & 1 & S_2 \\
      9{-}25  & {-}1001 & 2 & S_3 \\
      10{-}14 & 01{-}10 & 2 & S_4 \\
      25{-}29 & 11{-}01 & 3 & S_5 \\
      23{-}31 & 1{-}111 & 4 & S_6 \\
      29{-}31 & 111{-}1 & 4 & S_7 \\
      \hline
    \end{array}
    \]

  \item Paso 3: Omitimos este paso ya que con los resultados que tenemos no es posible hacer la comparación porque cambian en más de un bit.

  \item Hacemos la tabla para marcar lo que tenemos:
    \begin{table}[H]
      \begin{center}
        \begin{tabular}{| c | c | c | c | c | c | c | c | c | c | c |}
          
          \hline 
          & 0 & 2 & 9 & 10 & 14 & 25 & 23 & 29 & 31   \\ \hline
          $S_1$ & \cellcolor{mostaza}\ekis & \cellcolor{mostaza}\ekis & & & & & & & \\ \hline
          $S_2$ & & \ekis & & \ekis & & & & & \\ \hline
          $S_3$ & & & \cellcolor{mostaza}\ekis & & & \cellcolor{mostaza}\ekis & & & \\ \hline
          $S_4$ & & & & \cellcolor{mostaza}\ekis & \cellcolor{mostaza}\ekis & & & & \\ \hline
          $S_5$ & & & & & & \ekis & & \ekis & \\ \hline
          $S_6$ & & & & & & & \cellcolor{mostaza}\ekis & & \cellcolor{mostaza}\ekis \\ \hline
          $S_7$ & & & & & & & & \cellcolor{mostaza}\ekis & \cellcolor{mostaza}\ekis \\ \hline
        \end{tabular}
      \end{center}
    \end{table}
  \end{enumerate}
  
  Elegimos soluciones: $S_1,\; S_3,\; S_4,\; S_6,\; S_7$

  \[
  \therefore \text{La \textbf{Funci\'{o}n simplificada} es:}
  \]
  \[
  F(x_0, x_1, x_2, x_3, x_4) = \overline{x_0x_1x_2x_4} + x_1\overline{x_2x_3}x_4 + \overline{x_0}x_1x_2\overline{x_4} + x_0x_2x_3x_4 + x_0x_1x_2x_4
  \]
  \bigskip
  
\item Utilizando el algoritmo de Quine–McCluskey realiza la siguiente reducción.
  \[
  F(x_0, x_1, x_2, x_3, x_4) = \overline{x_0x_1x_2x_3x_4}\; +\; \overline{x_0x_1x_2}x_3\overline{x_4}\; +\; \overline{x_0x_1}x_2x_3\overline{x_4}\; +\; \overline{x_0x_1}x_2x_3x_4\; +
  \]
  \[
  \overline{x_0}x_1x_2x_3\overline{x_4}\; +\; \overline{x_0}x_1\overline{x_2}x_3\overline{x_4}\; +\; x_0\overline{x_1}x_2x_3x_4\; +\; x_0x_1\overline{x_2x_3}x_4\; +\; \overline{x_0}x_1\overline{x_2x_3}x_4\; +
  \]
  \[
  x_0x_1x_2\overline{x_3}x_4\; +\;  x_0x_1x_2x_3x_4
  \]
  % -- Respuesta -- %

  \begin{enumerate}[label=\arabic*)]
  \item Paso 1: Hacer la tabla y organizar según su índice.
    \[
    \begin{array}{| c | c | c |}
      \hline
      \text{Número} & x_0\,x_1\,x_2\,x_3\,x_4 & \text{Índice} \\
      \hline
      0  & 00000 & 0 \\
      2  & 00010 & 1 \\
      6  & 00110 & 2 \\
      9  & 01001 & 2 \\
      10 & 01010 & 2 \\
      7  & 00111 & 3 \\
      14 & 01110 & 3 \\
      25 & 11001 & 3 \\
      23 & 10111 & 4 \\
      29 & 11101 & 4 \\
      31 & 11111 & 5 \\
      \hline
    \end{array}
    \]
    
  \item Paso 2: Identificar cuáles cambian en un solo bit.
    \[
    \begin{array}{| c | c | c | c |}
      \hline
      \text{Combinación} & x_0\,x_1\,x_2\,x_3\,x_4 & \text{Índice} & \text{Solución} \\
      \hline
      0{-}2   & 000{-}0 & 0 & S_1 \\
      2{-}6   & 00{-}10 & 1 & X \\
      2{-}10  & 0{-}010 & 1 & X \\
      6{-}7   & 0011{-} & 2 & S_2 \\
      6{-}14  & 0{-}110 & 2 & X \\
      9{-}25  & {-}1001 & 2 & S_3 \\
      10{-}14 & 01{-}10 & 2 & X \\
      7{-}23  & {-}0111 & 3 & S_4 \\
      25{-}29 & 11{-}01 & 3 & S_5 \\
      23{-}31 & 1{-}111 & 4 & S_6 \\
      29{-}31 & 111{-}1 & 4 & S_7 \\
      \hline
    \end{array}
    \]

  \item Paso 3: Comparamos de nuevo los bits pero si el guión está en la misma posición.
    \[
    \begin{array}{| c | c | c | c |}
      \hline
      \text{Combinación} & x_0\,x_1\,x_2\,x_3\,x_4 & \text{Índice} & \text{Solución} \\
      \hline
      2,6{-}10,14   & 0{-}{-}10 & 1 & S_8 \\
      2,10{-}6,14   & 0{-}{-}10 & 1 & S_8\\
      \hline
    \end{array}
    \]

  \item Hacemos la tabla para marcar lo que tenemos:
    \begin{table}[H]
      \begin{center}
        \begin{tabular}{| c | c | c | c | c | c | c | c | c | c | c | c |}
          
          \hline 
          & 0 & 2 & 6 & 7 & 9 & 10 & 14 & 23 & 25 & 29 & 31   \\ \hline
          $S_1$ & \cellcolor{mostaza}\ekis & \cellcolor{mostaza}\ekis & & & & & & & & & \\ \hline
          $S_2$ & & & \ekis & \ekis & & & & & & & \\ \hline
          $S_3$ & & & & & \cellcolor{mostaza}\ekis & & & & \cellcolor{mostaza}\ekis & & \\ \hline
          $S_4$ & & & & \cellcolor{mostaza}\ekis & & & & \cellcolor{mostaza}\ekis & & & \\ \hline
          $S_5$ & & & & & & & & & \ekis & \ekis & \\ \hline
          $S_6$ & & & & & & & & \ekis & & & \ekis \\ \hline
          $S_7$ & & & & & & & & & & \cellcolor{mostaza}\ekis & \cellcolor{mostaza}\ekis \\ \hline
          $S_8$ & & \cellcolor{mostaza}\ekis & \cellcolor{mostaza}\ekis & & & \cellcolor{mostaza}\ekis & \cellcolor{mostaza}\ekis & & & & \\ \hline
        \end{tabular}
      \end{center}
    \end{table}
  \end{enumerate}
  
  Elegimos soluciones: $S_1,\; S_3,\; S_4,\; S_7,\; S_8$

  \[
  \therefore \text{La \textbf{Funci\'{o}n simplificada} es:}
  \]
  \[
  F(x_0, x_1, x_2, x_3, x_4) = \overline{x_0x_1x_2x_4} + x_1\overline{x_2x_3}x_4 + \overline{x_1}x_2x_3x_4 + x_0x_1x_2x_4 + \overline{x_0}x_3\overline{x_4}
  \]
\end{enumerate}

\begin{thebibliography}{1}

\bibitem{ejercicio1} 
  Galaviz, C. \textbf{(s. f.)}. \textit{Lógica digital y diseño de circuitos digitales}. [Facultad de Ciencias, UNAM]. PDF. Disponible en:

  \url{https://drive.google.com/file/d/1BdCwuwFcSar5W5nPxA98FVNfTxZfd6yg/view}
\end{thebibliography}
\end{document}
