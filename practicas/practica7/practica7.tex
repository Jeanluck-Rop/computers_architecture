\documentclass[12pt,letterpaper]{article}
\usepackage[utf8]{inputenc}
\usepackage{listings, float, xcolor}

%----- Configuración del estilo del documento------%
\usepackage{graphicx, fancyhdr}
\usepackage{enumitem, pifont, hyperref, ulem, tabularx}
\usepackage[left=2cm,right=2cm,top=1.8cm,bottom=2.3cm]{geometry}
\usepackage{hyperref}
\usepackage{lastpage}

%------ Paquetes matemáticos básicos --------%
\usepackage{amsmath, amssymb, amsthm}

%------ Definimos los colores para la sintaxis del código --------%
\definecolor{keywordcolor}{rgb}{0.5, 0.0, 0.5} % Azul para palabras clave
\definecolor{commentcolor}{rgb}{0.25, 0.5, 0.35} % Verde para comentarios
\definecolor{stringcolor}{rgb}{0.58, 0.0, 0.82} % Púrpura para strings

%------ Configuración para mostrar código VHDL  --------%
\lstdefinestyle{cppstyle}{
    language=c++,
    basicstyle=\ttfamily\footnotesize,
    keywordstyle=\color{keywordcolor}\bfseries,
    commentstyle=\color{commentcolor},
    stringstyle=\color{stringcolor},
    numbers=left,
    numberstyle=\tiny,
    stepnumber=1,
    numbersep=5pt,
    backgroundcolor=\color{gray!10},
    tabsize=2,
    showspaces=false,
    showstringspaces=false,
    breaklines=true,
    frame=single,
    captionpos=b
}

\begin{document}

%------ Encabezado -------- %
\begin{center}
\newcommand{\imp}{\rightarrow}
\newcommand{\vp}{\varphi}
  \begin{minipage}{3cm}
    \begin{center}
      \includegraphics[height=3.4cm]{../unam_logo.png}
    \end{center}
  \end{minipage}\hfill
  \begin{minipage}{10cm}
    \begin{center}
      \textbf{\Large Universidad Nacional Autónoma de México}\\[0.2cm]
      \textbf{\large Facultad de Ciencias}\\[0.2cm]
      \textbf{Organización y Arquitectura de Computadoras 2025-2}\\[0.4cm]
      \textbf{\Large Práctica 07}\\[0.1cm]
      \textbf{Docentes:}\\
      José Galaviz \hspace{1em} Ricardo Pérez \hspace{1em} Ximena Lezama\\[0.3cm]
      \textbf{Autores:}\\
      Fernanda Ramírez Juárez \quad Ianluck Rojo Peña\\[0.3cm]
      \textbf{Fecha de entrega:} Jueves 3 de abril de 2025
    \end{center}
  \end{minipage}\hfill
  \begin{minipage}{3cm}
    \begin{center}
      \includegraphics[height=3.4cm]{../fc_logo.png}
    \end{center}
  \end{minipage}
\end{center}

\bigskip
\hrule height 0.1pt
\bigskip

%------ Contenido -------- %
\section*{Ejercicios.}

\begin{lstlisting}[style=cppstyle, caption={Código en C++ - Calculador de Serie}]
  
\end{lstlisting}

\section*{Preguntas.}

\begin{enumerate}
\item En el ejercicio 1:\\
  \bigskip
  % -- Respuesta -- %
  \begin{itemize}
  \item ¿A qué valor tiende la serie?
    
    
  \item ¿A cuántos dígitos estaría limitado nuestro resultado con precisión sencilla? (Flotante) Justifica tu respuesta.
    
    
  \item ¿Cuántas iteraciones son necesarias para calcular el mayor número de dígitos?
    
    
  \end{itemize}
  \bigskip
  
\item Cuando tenemos un número doble guardado a lo largo de 2 registros, ¿Qué datos guarda cada registro?
  \bigskip
  % -- Respuesta -- %
  
  \bigskip
  
\item En MARS, en la barra de herramientas, en la pestaña de Tools, existe la herramienta llamada MIPS X-Ray, conecta esta herramienta y corre un programa línea por línea. ¿Qué significan los números resaltados de color magenta, verde, azul y azul claro que se encuentran abajo de la instrucción?
  \bigskip
  % -- Respuesta -- %
  
  \bigskip
  
\end{enumerate}
\end{document}
