\documentclass[12pt,letterpaper]{article}
\usepackage[utf8]{inputenc}
\usepackage{listings, float, xcolor}

%----- Configuración del estilo del documento------%
\usepackage{graphicx, fancyhdr}
\usepackage{enumitem, pifont, hyperref, ulem, tabularx}
\usepackage[left=2cm,right=2cm,top=1.8cm,bottom=2.3cm]{geometry}


%------ Paquetes matemáticos básicos --------%
\usepackage{amsmath, amssymb, amsthm}

\renewcommand{\lstlistingname}{Código}

%------ Definimos los colores para la sintaxis del código --------%
\definecolor{keywordcolor}{rgb}{0.5, 0.0, 0.5}  % Morado para palabras clave
\definecolor{commentcolor}{rgb}{0.25, 0.5, 0.35} % Verde para comentarios
\definecolor{stringcolor}{rgb}{0.88, 0.68, 0.18}  % Mostaza anaranjado para los strings
\definecolor{backgroundcolor}{rgb}{0.95, 0.95, 0.95} % Gris claro para fondo

%------ Configuración para mostrar código en C++ --------%
\lstdefinestyle{cppstyle}{
  language=C++,
  basicstyle=\ttfamily\footnotesize,
  keywordstyle=\color{keywordcolor}\bfseries,
  commentstyle=\color{commentcolor},
  stringstyle=\color{stringcolor},
  numbers=left,
  numberstyle=\tiny,
  stepnumber=1,
  numbersep=8pt,
  backgroundcolor=\color{backgroundcolor},
  tabsize=2,
  showspaces=false,
  showstringspaces=false,
  breaklines=true,
  frame=single,
  captionpos=b
}

\begin{document}

%------ Encabezado -------- %
\hrule height 0.1pt
\bigskip
\begin{center}
  \begin{minipage}{3cm}
    \begin{center}
      \includegraphics[height=3.4cm]{../unam_logo.png}
    \end{center}
  \end{minipage}\hfill
  \begin{minipage}{10cm}
    \begin{center}
      \textbf{\Large Universidad Nacional Autónoma de México}\\[0.2cm]
      \textbf{\large Facultad de Ciencias}\\[0.2cm]
      \textbf{Organización y Arquitectura de Computadoras 2025-2}\\[0.4cm]
      \textbf{\Large Práctica 09}\\[0.1cm]
      \textbf{Docentes:}\\
      José Galaviz \hspace{1em} Ricardo Pérez \hspace{1em} Ximena Lezama\\[0.3cm]
      \textbf{Autores:}\\
      Fernanda Ramírez Juárez \quad Ianluck Rojo Peña\\[0.3cm]
      \textbf{Fecha de entrega:} Jueves 9 de abril de 2026
    \end{center}
  \end{minipage}\hfill
  \begin{minipage}{3cm}
    \begin{center}
      \includegraphics[height=3.4cm]{../fc_logo.png}
    \end{center}
  \end{minipage}
\end{center}

\bigskip
\hrule height 0.1pt
\bigskip

%------ Contenido -------- %
\section*{Ejercicios.}

\newpage

\section*{Preguntas.}

\begin{enumerate}
\item En las llamadas a sistema, las bibliotecas y APIs funcionan como in-
termediario entre el usuario y las llamadas a sistema. ¿Que bibliotecas

contienen esas llamadas a sistema en Unix y en Windows? ¿Que llamadas
al sistema est ́an contenidas en esos archivos?
  % -- Respuesta -- %
  
  \bigskip
  
\item Describe que causa, que llamada a sistema y que es s o que hacen las
siguientes vulnerabilidades a Sistema causadas por la llamadas a Sistema
en Linux:1
Dirty Cow
Dirty Pipe
Baron Samedit Sudo
Residual Risk Flaw
  % -- Respuesta -- %
  
\end{enumerate}
\end{document}
